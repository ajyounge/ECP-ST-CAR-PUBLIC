\subsubsection{\stid{2.01} \tools\ Software Development Kits} 

\paragraph{Overview}
The Software Development Tools SDK is a collection of independent projects specifically targeted to address performance analysis at scale. The primary responsibility of the SDK is to coordinate the disparate development, testing, and deployment activities of many individual projects to produce a unified set of tools ready for use on the upcoming exascale machines. The efforts in support of the SDK are designed to fit within the overarching goal to leverage and integrate data measurement, acquisition, storage, analysis, and visualization techniques being developed across the ECP Software Technology ecosystem.


\paragraph{Key Challenges}
In addition to the general challenges faced by all of the SDKs outlined in Section~\ref{subsubsect:ecosystem-sdk}, the unique position of the \tools\ SDK between the hardware teams and the application developers requires additional effort in preparing today’s software to run on yet-unknown architectures and runtimes to be delivered by the end of ECP.

\paragraph{Solution Strategy}
The primary mechanism for mitigating risk in the SDK is the \textit{Readiness Survey}. This survey is designed to assess the current status of each product in the SDK in six key areas: software availability, documentation, testing, Spack build support, SDK integration, and path forward technology utilization. By periodically assessing the progress of the individual L4 products in the SDK, we will use the survey to identify and resolve current hardware architecture dependencies, plan for future architecture changes, and increase adoption of the Continuous Integration (CI) testing workflow to reduce this risk.

Critically, the survey will allow us to accomplish this by providing a direct communication channel between the SDK maintainers and the L4 product developers allowing us to identify current architecture dependencies in each project and compare them with existing and emerging ECP platforms. Our initial efforts will be to increase support for today’s heterogeneous CPU architectures across the DOE facilities (e.g., x86, Power, ARM, etc.) to ensure a minimum level of usability on these platforms. We will then focus on current accelerator architectures- namely GPGPU computing. As new architectures arise, we will re-issue the survey and use this same process to provide guidance to the L4 product as they develop support for them.

The survey also allows us to monitor the increased adoption of the proposed ECP CI testing workflow. This will be crucial to understanding each project’s interoperability with not only the other projects within the Tools SDK, but all applications across the ECP Software Technologies landscape. Additionally, it will serve as a bridge between the Hardware Integration teams working with the facilities and the software teams working across the SDK. By relaying new hardware requirements from the facilities to the software developers, we can closely monitor support for both new and existing systems. Conversely, giving feedback to the facilities regarding compiler support and buildability of library dependencies will guide software adoption on those platforms.

\paragraph{Recent Progress}
The Readiness Survey was presented to the Principle Investigator of each L4 product in May 2019, and the results were tabulated in July 2019. Overall, the products show very good coverage of four of the six areas. In particular, we found that all of the products in the SDK have working Spack packages. This is a very significant finding as it allows us to quickly move ahead to focus on the remaining two areas, testing and GPU support, which need more attention.

In addition to the survey, the Tools SDK was the first ECP project to carry out the integration of an SDK product into the Gitlab Continuous Integration testing workflow proposed by the ECP Hardware and Integration team. Using the GitLab source code repository hosted at the U.S. Department of Energy Office of Scientific and Technical Information and computing resources at the New Mexico Consortium, we successfully demonstrated the essential functionality of the CI workflow. Although there is still much to be done, this represents a substantial milestone in a core component of the sustainability initiative for ECP.


\paragraph{Next Steps}
Although all of the L4 products in the SDK have Spack packages, only six are in the first release of the E4S distribution. Our first step is to get the remaining Spack packages tested and integrated into E4S to further our KPP3 goal. Additional testing using multiple compilers- including some variant of LLVM currently in use by the Compilers and Debuggers SDK- on at least one current DOE facility machine will be carried out. Results from these tests will continue to be fed back into the L4 products to further guide development of spack packages, bug/issue-reporting workflows, and integration into the greater ECP software ecosystem. Any discovered issues with Spack, compilers, or libraries will be directly reported back to their respective development teams or L3 representative.

Increasing the number of L4 products in the SDK with CI testing adoption is our second goal. A pilot project using the Dyninst Binary Tools Suite was successfully carried out in FY2019. Using this as a foundation, we propose to include at least an another four of the L4 products into the CI pipeline. Arguably, establishing this workflow is the largest contribution the Tools SDK will bring to the overall ECP software ecosystem. Having automated testing in place across heterogeneous build environments and target architectures is a fundamental challenge to creating reliable, sustainable software- making this work a critical path to attaining the ECP goals of large-scale software sustainability. We also anticipate that this may be the introduction of formal software testing for some of the L4 products. The heterogeneous nature of the testing available in the Tools SDK L4 products will serve as a focused testbed for constructing implementation guidelines for the CI workflow which can then be applied across the SDK efforts and into the greater ECP software ecosystem. Importantly, these lessons can also be carried on by the individual project teams to help maintain their software beyond the ECP timeline.

