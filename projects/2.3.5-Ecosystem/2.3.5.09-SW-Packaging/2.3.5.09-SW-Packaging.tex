\subsubsection{\stid{5.09} Software Packaging Technologies} \label{subsubsect:sw-packaging}


\paragraph{Overview}

ECP is tasked with building the first capable exascale ecosystem, and the
foundation of this ecosystem, per ECP's mission statement, is {\it an
integrated software stack}.  Building and integrating software for
supercomputers is notoriously difficult, and an integration effort for
HPC software at this scale is unprecedented.  Moreover, the software
deployment landscape is changing as containers and supercomputing-capable
software package managers like Spack emerge.  Spack holds the promise to
automate the builds of all ECP software, and to allow it to be
distributed in new ways, including as binary packages.  Containers will
enable entire application deployments to be packaged into reproducible
images, and they hold the potential to accelerate development and
continuous integration (CI) workflows.

This project will build the tooling required to ensure that packaging
technologies can meet the demands of the ECP ecosystem.  The project
provides Spack packaging assistance for ST users and ECP facilities, and
it develops new capabilities for Spack that enable automated deployments
of software at ECP facilities, in containerized environments, and as part
of continuous integration.  Concurrently, the ``Supercontainers''
sub-project is investigating and developing technologies and best
practices that enable containers to be used effectively at ECP
facilities. Supercontainers will ensure that HPC container runtimes will
be scalable, interoperable, and integrated into Exascale supercomputing
across DOE.


\paragraph{Key Challenges}

Historically, building software to run as fast as possible on HPC
machines has been a manual process.  Users download source code for
packages they wish to install, and they build and tune it manually for
high performance machines. Spack has automated much of this process, but
it still requires that users {\it build} software.  Spack needs
modifications to enable it to understand complex microarchitecture
details, ABI constraints, and runtime details of exascale machines.  This
project will enable binary packaging, and it will develop new
technologies that enable the same binary packages to to be used within
containers {\it or} in bare metal deployments on exascale hardware.

The Supercontainer effort faces similar challenges to deploying
containers on HPC machines. Container technology most notably enables
users to define their own software environments, using all the facilities
of the containerized host OS.  Users can essentially bring {\it their
own} software stack to HPC systems, and they can snapshot an entire
application deployment, including dependencies, within a container.
Containers also offer the potential for portability between users and
machines. The goal of moving an HPC application container from a laptop
to a supercomputer with little or no modification is in reach, but there
are a number of challenges to overcome before this is possible on
Exascale machines.  Solutions from industry, such as Docker, assume that
containers can be built and run with elevated privileges, that containers
are isolated from the host network, filesystem, and GPU hardware, and
that binaries within a container are unoptimized and can run on any chip
generation from a particular architecture. These go against the
multi-user, multi-tenant user environment of most HPC centers, and
optimized containers may not be portable across systems.

\paragraph{Solution Strategy}

The Spack project supports ST teams by developing portable build recipes
and additional metadata for the ECP package ecosystem.  The end goal is
to provide a packaging solution that can deploy on bare metal, in a
container, or be {\it rebuilt} for a new machine on demand. Spack bridges
the portability divide with portable package recipes; specialized
packages can be built per-site if needed, or lowest-common denominator
packages can be built for those cases that do not need highly optimized
performance.  Packages are relocatable and can be used outside their
original build environment.  Moreover, Spack provides {\it environments}
that enable a number of software packages to be deployed together either
on an HPC system or in a container.

The Supercontainer project seeks to document current practice and to
leverage existing container runtimes, but also to develop new enabling
technologies where necessary to allow containers to run on HPC machines.
Several HPC container runtimes (Shifter, Charliecloud, and Singularity)
already exist, and this diversity enables wide exploration of the HPC
container design space, and the Supercontainers project will work with
their developers to address HPC-specific needs, including container and
job launch, resource manager integration, distribution of images at
scale, use of system hardware (storage systems, network and MPI
libraries, GPUs and other accelerators), and usability concerns around
interfacing between the host and container OS (e.g., bind-mounting,
etc. required for hardware support).

The project will document best practices and produce a technical report
to help educate new users and developers to the advantages of containers,
as well as a best-practices report to help ensure efficient container
utilization on supercomputers. Both of these will be living
documents, periodically updated in response to lessons learned and
feedback.  In addition, we will identify gaps, and implement changes and
automation in one of the three existing runtimes, as needed.  The project
will also interface with the E4S and SDK teams, as well as AD teams
interested in containerizing their applications. We will work to enable
these teams to deploy reproducible, minimally-sized container images that
support multiple AD software ecosystems.

\paragraph{Recent Progress}

\begin{enumerate}
\item Released {\tt archspec}\footnote{https://github.com/archspec/archspec/},
      a library for lableing and distributing CPU microarchitectures, as a
      spin-off project from Spack.

\item Developed the {\tt spack containerize} command that enables users to
      easily build lightweight containers from Spack environments.

\item Worked with the E4S team to get E4S packages building on the Ascent
      machine at OLCF, onNERSC's Cori machine, and in continuous integration in
      Amazon Web Services (on optimized {\tt skylake} instances).

\item Continued to support ECP ST teams and conducted a survey to better
      understand their usage of Spack.

\item Developed optimized E4S container images for {\tt ppc64le},
      {\tt x86\_64}, as well as CUDA and ROCm GPUs. Solutions for
      heterogeneous hardware usage via containers are ahead of available
      solutions in industry. \footnote{https://hub.docker.com/u/ecpe4s}.

\item Continued to test multiple container runtimes on pre-exascale systems.
\end{enumerate}

\paragraph{Next Steps}

\begin{enumerate}
\item Implement support for running test suites within Spack environments, and
      develop ways for teams to build portable tests associated with Spack
      packages (``{\tt spack test}'').

\item Integrate {\tt spack test} with GitLab continuous integration to display
      dashboards for E4S tests run on EA systems.

\item Work with the E4S team to implement initial package smoke tests for E4S
      on EA systems.

\item Improve automation on GitHub by implementing a change notification bot
      for Spack package maintainers, to tell them when others have proposed
      changes to their packages.

\item Implement pull request testing for Spack GitHub, so that every pull
      request to Spack has its package builds tested in a sandboxed
      environment to ensure that the {\tt develop} branch continues to build.

\item Investigate unprivileged container builds at ECP facilities and determine
      which runtimes provide the best support for this.

\item Continue container training sessions, outreach efforts, and tutorials.
\end{enumerate}
